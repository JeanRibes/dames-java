%%%%%%%%%%%%%%%%%%%%%%%%%%%%%%%%%%%%%%%%%
% University/School Laboratory Report
% LaTeX Template
% Version 3.1 (25/3/14)
%
% This template has been downloaded from:
% http://www.LaTeXTemplates.com
%
% Original author:
% Linux and Unix Users Group at Virginia Tech Wiki 
% (https://vtluug.org/wiki/Example_LaTeX_chem_lab_report)
%
% License:
% CC BY-NC-SA 3.0 (http://creativecommons.org/licenses/by-nc-sa/3.0/)
%
%%%%%%%%%%%%%%%%%%%%%%%%%%%%%%%%%%%%%%%%%

%----------------------------------------------------------------------------------------
%	PACKAGES AND DOCUMENT CONFIGURATIONS
%----------------------------------------------------------------------------------------

\documentclass{article}

\usepackage[version=3]{mhchem} % Package for chemical equation typesetting
\usepackage{siunitx} % Provides the \SI{}{} and \si{} command for typesetting SI units
\usepackage{graphicx} % Required for the inclusion of images
\usepackage{natbib} % Required to change bibliography style to APA
\usepackage{amsmath} % Required for some math elements 
\usepackage[utf8]{inputenc}   
\usepackage[frenchb]{babel} 

\setlength\parindent{0pt} % Removes all indentation from paragraphs

\renewcommand{\labelenumi}{\alph{enumi}.} % Make numbering in the enumerate environment by letter rather than number (e.g. section 6)

%\usepackage{times} % Uncomment to use the Times New Roman font

%----------------------------------------------------------------------------------------
%	DOCUMENT INFORMATION
%----------------------------------------------------------------------------------------

\title{Projet de groupe :\\ Jeu de Dames} % Title

\author{Jean \textsc{Ribes}\\Bastian \textsc{NORROY}\\Aldwin \textsc{AUBRY}} % Author name

\date{} % Date for the report

\begin{document}

\maketitle % Insert the title, author and date
\vfill
%\begin{center}
%\begin{tabular}{l r}
%Date Performed: & January 1, 2012 \\ % Date the experiment was performed
%Partners: & James Smith \\ % Partner names
%& Mary Smith \\
%Instructor: & Professor Smith % Instructor/supervisor
%\end{tabular}
%\end{center}

% If you wish to include an abstract, uncomment the lines below
\begin{abstract}
Notre projet consiste à produire un jeu de dames sous Java en utilisant les notions acquises depuis le début de l'année. A savoir, l'usage de tableaux 2D, d'objets, d'affichage.
\end{abstract}
%----------------------------------------------------------------------------------------
%	SECTION 1
%----------------------------------------------------------------------------------------
\vfill
\section{Objectifs}

Nos objectifs sont les suivants :
\begin{itemize}
\item affichage du tour, quel joueur doit jouer\,;
\item vérifier le nombre d'actions possibles\,;
\item auto-jouer les coups obligatoire quand il n'y a pas d'autres choix\,;
\item affichage des coups permis (quels pions manger, où aller)\,;
\item Sélectionner un pion à l'aide d'un curseur\,;
\item calculer des scores.
\item Disposer d'un mode multijoueur, avec une synchronisation bloquante et fontionnant à travers internet
\end{itemize}
%\begin{center}==== POUR APRÈS ===\end{center}
\subsection{Pour après}
\begin{itemize}
\item Intégrer une IA (bien op pour que tu perdes d'office).
\end{itemize}
\newpage

\subsection{Multijoueur}
\begin{itemize}
  \item Avoir un "lobby" où les joueurs peuvent se rejoindre en jeu
  \item Gérer correctement la fin de la partie
  \item Communiquer rapidement, sans "lag" les informations du jeu à travers Internet.
  \item Utiliser un serveur central qui organise les parties et route les informations du jeu.
\end{itemize}
%\begin{center}\ce{==== POUR APRÈS ===}\end{center}

% If you have more than one objective, uncomment the below:
%\begin{description}
%\item[First Objective] \hfill \\
%Objective 1 text
%\item[Second Objective] \hfill \\
%Objective 2 text
%\end{description}


%----------------------------------------------------------------------------------------


\end{document}
